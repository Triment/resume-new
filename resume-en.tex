%%
%% Copyright (c) 2018-2019 Weitian LI <wt@liwt.net>
%% CC BY 4.0 License
%%
%% Résumé
%% ------
%% A short document (1-2 pages) to sum up the job-related accomplishments
%% and experience.
%%
%% Checklist
%% ---------
%% * Contact Information
%% * Work History / Experience
%% * Education
%% * Skills
%% * Summary & Objective (optional)
%% * Hobbies & Interests (optional)
%%
%% Credits
%% -------
%% * CV vs. Resume: What is the Difference? When to Use Which?
%%   https://uptowork.com/blog/cv-vs-resume-difference
%% * How to Make a Resume: A Step-by-Step Guide (+30 Examples)
%%   https://uptowork.com/blog/how-to-make-a-resume
%% * Entry-Level Resume: Sample and Complete Guide (+20 Examples)
%%   https://uptowork.com/blog/entry-level-resume-example
%%
%% Created: 2018-04-14
%%

% English version
\documentclass{resume}

% File information shown at the footer of the last page
\fileinfo{%
  \faCopyright{} 2018--2020, Weitian LI \hspace{0.5em}
  \creativecommons{by}{4.0} \hspace{0.5em}
  \githublink{liweitianux}{resume} \hspace{0.5em}
  \faEdit{} \today
}

\name{Weitian}{LI}

\keywords{BSD, Linux, Programming, Python, C, Shell, DevOps, SysAdmin}

% \tagline{\texorpdfstring{\icon{\faBinoculars} }{}<position-to-look-for>}
% \tagline{<current-position>}

% Supported shapes: circular (default), square
% \photo[<shape>]{<width>}{<filename>}

\profile{
  \mobile{132-6262-0332}
  \email{liweitianux@live.com}
  \github{liweitianux} \\
  \degree{Ph.D. in Physics}
  \university{Shanghai Jiao Tong University (SJTU)}
  \birthday{1991 Sept.}
  \address{Shanghai}
  % Custom information:
  % \icontext{<icon>}{<text>}
  % \iconlink{<icon>}{<link>}{<text>}
}

\begin{document}
\makeheader

%======================================================================
% Summary & Objectives
%======================================================================
Highly-motivated Ph.D. in Physics (radio astronomy)
with good foundations of math and statistics.
Proficient in data modeling and analysis,
and enthusiastic about computer and network technologies.
With 10 years experience in Linux and BSD,
skilled in Shell, Python, and C programming.
Passionate about open source and share multiple projects on my
\link{https://github.com/liweitianux}{GitHub}.
Meanwhile a \link{https://www.dragonflybsd.org}{DragonFly BSD}
operating system developer and a contributor to several other
open source projects.

%======================================================================
\sectionTitle{Competences \& Languages}{\faWrench}
%======================================================================
\begin{competences}[10em]
  \comptence{Operating Systems}{
    \icon{\faLinux} Linux (10 years),
    \icon{\faFreebsd} DragonFly BSD \& FreeBSD (7 years)
  }
  \comptence{Programming}{%
    Python, C, Shell, R, Tcl/Tk
  }
  \comptence{Tools}{%
    SSH, Git, Make, Tmux, Vi, Ansible
  }
  \comptence{Data Analysis}{%
    R, Pandas; Matplotlib, ggplot2; Keras, Scikit-learn
  }
  \comptence{Web Development}{%
    Flask, JavaScript, jQuery, Bootstrap
  }
  \comptence{\icon{\faLanguage} Languages}{
    \textbf{English} ---
      reading \& writing (good);
      listening \& speaking (conversant)
  }
\end{competences}

%======================================================================
\sectionTitle{Education}{\faGraduationCap}
%======================================================================
\begin{educations}
  \education%
    {September 2013}%
    [September 2019]%
    {Shanghai Jiao Tong University}%
    {School of Physics and Astronomy}%
    {Physics}%
    {Ph.D.}

  \separator{0.5ex}
  \education%
    {September 2009}%
    [June 2013]%
    {Shanghai Jiao Tong University}%
    {Department of Physics and Astronomy}%
    {Applied Physics}%
    {Bachelor's Degree}
\end{educations}

%======================================================================
\sectionTitle{Computer Skills}{\faCogs}
%======================================================================
\begin{itemize}
  \item DragonFly BSD operating system developer:
    200+ code commits; kernel and system utilities;
    participate in discussions and anwser questions
    in mailing lists and the IRC channel.
  \item Use Ansible to manage a VPS running DragonFly BSD that serves
    personal email, authoritative DNS, website, Git, IRC, etc.
  \item Built and administrate the workstations, a 4-node computer cluster,
    and network facilities for the team.
  \item Participated in building and testing the SKA high-performance
    cluster prototype (1 login node + 1 data node + 4 computing nodes)
    in Shanghai Astronomical Observatory.
  \item Designed and developed the whole website (Django, Bootstrap, jQuery)
    for \enquote{The 1st China--New Zealand Joint SKA Summer School}
    in 2014.
\end{itemize}

%======================================================================
\sectionTitle{Personal Projects}{\faCode}
%======================================================================
\begin{itemize}
  \item \link{https://github.com/liweitianux/atoolbox}{\texttt{atoolbox}}:
    (Python, Shell)
    Various tools collected over the years, to help manage systems,
    do daily tasks, analyze data, etc.
  \item \link{https://github.com/liweitianux/dfly-update}{\texttt{dfly-update}}:
    (Shell)
    A simple tool to update a DragonFly BSD system.
  \item \link{https://github.com/liweitianux/openrcs}{\texttt{openrcs}}:
    (C)
    Enhance \texttt{OpenBSD RCS}, to make it compatible with \texttt{GNU RCS}.
  \item \link{https://github.com/liweitianux/fg21sim}{\texttt{fg21sim}}:
    (Python)
    Simulate the low-frequency radio sky maps.
  \item \link{https://github.com/liweitianux/cdae-eor}{\texttt{cdae-eor}}:
    (Python, Keras)
    Use a Convolutional Denoising Autoencoder (CDAE) to separate the
    faint EoR signal.
  \item \link{https://github.com/liweitianux/chandra-acis-analysis}{\texttt{chandra-acis-analysis}}:
    (Python, Shell, Tcl)
    Semi-automate utilities for analyzing X-ray astronomical data.
  \item \link{https://github.com/liweitianux/resume}{\texttt{resume}}:
    (\LaTeX)
    The template and source files of \emph{this resume}.
\end{itemize}

%======================================================================
\sectionTitle{Research Achievements}{\faAtom}
%======================================================================
\begin{itemize}
  \item Developed the low-frequency radio sky image simulation software:
    \link{https://github.com/liweitianux/fg21sim}{\texttt{FG21sim}}.
  \item Developed a suite of utilities to semi-automate the
    X-ray astronomical data analysis:
    \link{https://github.com/liweitianux/chandra-acis-analysis}{\texttt{chandra-acis-analysis}}.
  \item Separated the faint cosmological EoR signal along the frequency
    dimension using a Convolutional Denoising Autoencoder (CDAE).
  \item Classified the radio galaxies in the FIRST survey according to
    morphologies using a Convolutional Neutral Network (CNN).
  \item Significantly improved the modeling of radio halos,
    and integrated the instrumental effects of radio interferometers
    into the simulation pipeline.
  \item Improved the background modeling in X-ray spectral fitting
    achieved more accurate and robust fitting results.
  \item Published 2 first-author and 8 co-authored SCI papers.
\end{itemize}

%======================================================================
\sectionTitle{Internships}{\faBriefcase}
%======================================================================
\begin{experiences}
  \experience%
    [April 2018]%
    {August 2018}%
    {Data Engineer @ Leadvisor Technology Inc. (startup company)}%
    [\begin{itemize}
      \item Search and scrape product and advertising data from Amazon web
        (Python, Requests, BeautifulSoup).
      \item Deployed the Airflow server and database to periodically
        retrieve product sales and advertising data from Amazon.
      \item Developed the website (Flask, jQuery) to help customers to
        optimize their advertising campaigns on Amazon.
    \end{itemize}]

  \separator{0.5ex}
  \experience%
    [July 2013]%
    {September 2013}%
    {Web Developer @ 97 Suifang (startup company)}%
    [\begin{itemize}
      \item Developed the back-end (Django) to support user registration,
        data storage and search.
      \item Developed the front-end (jQuery, AJAX) to visualize the
        temporal variations of a patient's examination indicators.
    \end{itemize}]
\end{experiences}

\end{document}
