%%
%% Copyright (c) 2018-2020 Weitian LI <wt@liwt.net>
%% CC BY 4.0 License
%%
%% Created: 2018-04-11
%%
% !TEX program = xelatex
% Chinese version
\documentclass[zh]{resume}

% File information shown at the footer of the last page
\fileinfo{%
  \faCopyright{} 2018--2020, Weitian LI \hspace{0.5em}
  \creativecommons{by}{4.0} \hspace{0.5em}
  \githublink{liweitianux}{resume} \hspace{0.5em}
  \faEdit{} \today
}

\name{帅}{林}

\keywords{BSD, Linux, Programming, Python, C, Shell, DevOps, SysAdmin}

% \tagline{\texorpdfstring{\icon{\faBinoculars} }{}<position-to-look-for>}
% \tagline{<current-position>}

% \photo[
%   shape=<circular|square>,    % default is circular
%   position=<left|right>,      % default is left
% ]{<width>}{<filename>}
% Example:
% \photo[shape=square]{5.5em}{photo}

\profile{
  \mobile{131-9889-8368}
  \email{13198898368@163.com}
  \github{Triment} \\
  \university{四川师范大学}
  \degree{小学教育 \textbullet 本科}
  \birthday{1995}
  \address{成都}
  % Custom information:
  % \icontext{<icon>}{<text>}
  % \iconlink{<icon>}{<link>}{<text>}
}

\begin{document}
\makeheader

%======================================================================
% Summary & Objectives
%======================================================================
% \begin{abstract}
% 物理学专业(射电天文方向)直博研究生,有扎实的物理、数学与统计学基础,
% 擅长数据建模与分析,热衷计算机和网络技术,
% 有 10 年的 Linux 和 BSD 使用经验,熟练掌握 Shell、Python 和 C 语言编程。
% 积极实践自由开源精神,
% 在 \link{https://github.com/liweitianux}{GitHub} 上分享多个项目,
% 是 \link{https://www.dragonflybsd.org}{DragonFly BSD} 操作系统的开发者,
% 并积极参与其他多个开源项目。
% \end{abstract}

%======================================================================
\sectionTitle{技能和语言}{\faWrench}
%======================================================================
\begin{competences}
  \comptence{操作系统}{%
    \icon{\faLinux} Linux (10 年),
     \& \icon{\faApple} MacOS (7 年)
  }
  \comptence{编程}{%
    Python, C, Shell, Rust, Node.js,Golang
  }
  \comptence{工具}{%
    SSH, Git
  }
  \comptence{网站开发}{%
    koa2, react, axum, SQL
  }
\end{competences}

%======================================================================
\sectionTitle{教育背景}{\faGraduationCap}
%======================================================================
\begin{educations}
  \education%
    {}%
    [2020]%
    {成都信息工程}%
    {物理与天文学院}%
    {计算机应用技术}%
    {专科}

  \separator{0.5ex}
  \education%
    {}%
    [2025]%
    {四川师范大学}%
    {小学教育}%
    {本科}
\end{educations}

%======================================================================
\sectionTitle{计算机技能}{\faCogs}
%======================================================================
\begin{itemize}
  \item Web网站开发:前后端分离,同构
  \item Linux 系统管理,网络设备配置,Proxmox虚拟平台,数通企业网络配置管理
  \item docker容器管理,数据库管理
  \item 设计并开发过微信三方登录应用
  \item AI大模型部署,系统提示注入
\end{itemize}

%======================================================================
\sectionTitle{个人项目}{\faCode}
%======================================================================
\begin{itemize}
  \item \link{https://github.com/Triment/eject}{\texttt{atoolbox}}:
    (Glang)
    基于前缀树的路由框架
  \item \link{https://github.com/Triment/react-markdown-editor}{\texttt{dfly-update}}:
    (react)
    支持Latex的Markdown编辑器
  \item \link{https://github.com/Triment/oidc}{\texttt{openrcs}}:
    (Rust)
    OIDC Provider标准服务器,提供单点登录,权限控制
  \item \link{https://github.com/Triment/proxy}{\texttt{fg21sim}}:
    (Rust)
    类似ngrok的HTTP域名转发代理服务器
  \item \link{https://github.com/Triment/exeo}{\texttt{cdae-eor}}:
    (Rust,React)
    设计的傻瓜式表格处理工具,利用worker批量处理数据
\end{itemize}

% %======================================================================
% \sectionTitle{工作经历}{\faIdCard}
% %======================================================================
% \begin{itemize}
%   \item 开发低频射电天空图像模拟软件:
%     \link{https://github.com/liweitianux/fg21sim}{\texttt{FG21sim}}
%   \item 开发程序实现~X~射线天文观测数据的半自动化分析:
%     \link{https://github.com/liweitianux/chandra-acis-analysis}{\texttt{chandra-acis-analysis}}
%   \item 利用卷积去噪自动编码器(CDAE)在频率维度分离微弱的宇宙再电离(EoR)信号
%   \item 利用卷积神经网络(CNN)对 FIRST 巡天的射电星系图像根据形态特征进行分类
%   \item 显著改进星系团射电晕的建模,并考虑低频干涉阵列的复杂仪器效应
%   \item 改进~X~射线光谱拟合的背景成分建模,获到更准确可靠的拟合结果
%   \item 发表 2 篇第一作者以及 8 篇合作者 SCI 论文
% \end{itemize}

% %======================================================================
\sectionTitle{工作经历}{\faIdCard}
%======================================================================
\begin{experiences}
  \experience%
    [2020.03]%
    {至今}%
    {省区IT@上海壹米滴答快运有限公司成都分公司}%
    [\begin{itemize}
      \item 处理内部信息化系统异常、挖掘需求、配合产品组提升产品服务质量
      \item 网络设备配置,管理,制定网络路由策略,NAT映射,负载均衡,子网性能优化
      \item 制作各种效率提升工具,脚本编写,自动化部署
      \item 网点系统培训支持
      \item 2021年获优秀IT支持组负责人
    \end{itemize}]

  \separator{0.5ex}
  \experience%
    [2018]%
    {2019}%
    {教研老师 @ 酷爱创意科技有限公司}%
    [\begin{itemize}
      \item C++编程,乐高EV3课程开发,集训相关资料制作。
      \item C数据结构和算法课程授课,集训带队参加NOIP竞赛。
      \item 研究探索Golang在EV3上的应用,与EV3教师合作发行行业标准化课程教材。
    \end{itemize}]
\end{experiences}

\end{document}
